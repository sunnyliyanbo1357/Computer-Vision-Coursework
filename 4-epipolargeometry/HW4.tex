
% Default to the notebook output style

    


% Inherit from the specified cell style.




    
\documentclass{article}

    
    
    \usepackage{graphicx} % Used to insert images
    \usepackage{adjustbox} % Used to constrain images to a maximum size 
    \usepackage{color} % Allow colors to be defined
    \usepackage{enumerate} % Needed for markdown enumerations to work
    \usepackage{geometry} % Used to adjust the document margins
    \usepackage{amsmath} % Equations
    \usepackage{amssymb} % Equations
    \usepackage{eurosym} % defines \euro
    \usepackage[mathletters]{ucs} % Extended unicode (utf-8) support
    \usepackage[utf8x]{inputenc} % Allow utf-8 characters in the tex document
    \usepackage{fancyvrb} % verbatim replacement that allows latex
    \usepackage{grffile} % extends the file name processing of package graphics 
                         % to support a larger range 
    % The hyperref package gives us a pdf with properly built
    % internal navigation ('pdf bookmarks' for the table of contents,
    % internal cross-reference links, web links for URLs, etc.)
    \usepackage{hyperref}
    \usepackage{longtable} % longtable support required by pandoc >1.10
    \usepackage{booktabs}  % table support for pandoc > 1.12.2
    

    
    
    \definecolor{orange}{cmyk}{0,0.4,0.8,0.2}
    \definecolor{darkorange}{rgb}{.71,0.21,0.01}
    \definecolor{darkgreen}{rgb}{.12,.54,.11}
    \definecolor{myteal}{rgb}{.26, .44, .56}
    \definecolor{gray}{gray}{0.45}
    \definecolor{lightgray}{gray}{.95}
    \definecolor{mediumgray}{gray}{.8}
    \definecolor{inputbackground}{rgb}{.95, .95, .85}
    \definecolor{outputbackground}{rgb}{.95, .95, .95}
    \definecolor{traceback}{rgb}{1, .95, .95}
    % ansi colors
    \definecolor{red}{rgb}{.6,0,0}
    \definecolor{green}{rgb}{0,.65,0}
    \definecolor{brown}{rgb}{0.6,0.6,0}
    \definecolor{blue}{rgb}{0,.145,.698}
    \definecolor{purple}{rgb}{.698,.145,.698}
    \definecolor{cyan}{rgb}{0,.698,.698}
    \definecolor{lightgray}{gray}{0.5}
    
    % bright ansi colors
    \definecolor{darkgray}{gray}{0.25}
    \definecolor{lightred}{rgb}{1.0,0.39,0.28}
    \definecolor{lightgreen}{rgb}{0.48,0.99,0.0}
    \definecolor{lightblue}{rgb}{0.53,0.81,0.92}
    \definecolor{lightpurple}{rgb}{0.87,0.63,0.87}
    \definecolor{lightcyan}{rgb}{0.5,1.0,0.83}
    
    % commands and environments needed by pandoc snippets
    % extracted from the output of `pandoc -s`
    \providecommand{\tightlist}{%
      \setlength{\itemsep}{0pt}\setlength{\parskip}{0pt}}
    \DefineVerbatimEnvironment{Highlighting}{Verbatim}{commandchars=\\\{\}}
    % Add ',fontsize=\small' for more characters per line
    \newenvironment{Shaded}{}{}
    \newcommand{\KeywordTok}[1]{\textcolor[rgb]{0.00,0.44,0.13}{\textbf{{#1}}}}
    \newcommand{\DataTypeTok}[1]{\textcolor[rgb]{0.56,0.13,0.00}{{#1}}}
    \newcommand{\DecValTok}[1]{\textcolor[rgb]{0.25,0.63,0.44}{{#1}}}
    \newcommand{\BaseNTok}[1]{\textcolor[rgb]{0.25,0.63,0.44}{{#1}}}
    \newcommand{\FloatTok}[1]{\textcolor[rgb]{0.25,0.63,0.44}{{#1}}}
    \newcommand{\CharTok}[1]{\textcolor[rgb]{0.25,0.44,0.63}{{#1}}}
    \newcommand{\StringTok}[1]{\textcolor[rgb]{0.25,0.44,0.63}{{#1}}}
    \newcommand{\CommentTok}[1]{\textcolor[rgb]{0.38,0.63,0.69}{\textit{{#1}}}}
    \newcommand{\OtherTok}[1]{\textcolor[rgb]{0.00,0.44,0.13}{{#1}}}
    \newcommand{\AlertTok}[1]{\textcolor[rgb]{1.00,0.00,0.00}{\textbf{{#1}}}}
    \newcommand{\FunctionTok}[1]{\textcolor[rgb]{0.02,0.16,0.49}{{#1}}}
    \newcommand{\RegionMarkerTok}[1]{{#1}}
    \newcommand{\ErrorTok}[1]{\textcolor[rgb]{1.00,0.00,0.00}{\textbf{{#1}}}}
    \newcommand{\NormalTok}[1]{{#1}}
    
    % Define a nice break command that doesn't care if a line doesn't already
    % exist.
    \def\br{\hspace*{\fill} \\* }
    % Math Jax compatability definitions
    \def\gt{>}
    \def\lt{<}
    % Document parameters
    \title{HW4}
    
    
    

    % Pygments definitions
    
\makeatletter
\def\PY@reset{\let\PY@it=\relax \let\PY@bf=\relax%
    \let\PY@ul=\relax \let\PY@tc=\relax%
    \let\PY@bc=\relax \let\PY@ff=\relax}
\def\PY@tok#1{\csname PY@tok@#1\endcsname}
\def\PY@toks#1+{\ifx\relax#1\empty\else%
    \PY@tok{#1}\expandafter\PY@toks\fi}
\def\PY@do#1{\PY@bc{\PY@tc{\PY@ul{%
    \PY@it{\PY@bf{\PY@ff{#1}}}}}}}
\def\PY#1#2{\PY@reset\PY@toks#1+\relax+\PY@do{#2}}

\expandafter\def\csname PY@tok@gd\endcsname{\def\PY@tc##1{\textcolor[rgb]{0.63,0.00,0.00}{##1}}}
\expandafter\def\csname PY@tok@gu\endcsname{\let\PY@bf=\textbf\def\PY@tc##1{\textcolor[rgb]{0.50,0.00,0.50}{##1}}}
\expandafter\def\csname PY@tok@gt\endcsname{\def\PY@tc##1{\textcolor[rgb]{0.00,0.27,0.87}{##1}}}
\expandafter\def\csname PY@tok@gs\endcsname{\let\PY@bf=\textbf}
\expandafter\def\csname PY@tok@gr\endcsname{\def\PY@tc##1{\textcolor[rgb]{1.00,0.00,0.00}{##1}}}
\expandafter\def\csname PY@tok@cm\endcsname{\let\PY@it=\textit\def\PY@tc##1{\textcolor[rgb]{0.25,0.50,0.50}{##1}}}
\expandafter\def\csname PY@tok@vg\endcsname{\def\PY@tc##1{\textcolor[rgb]{0.10,0.09,0.49}{##1}}}
\expandafter\def\csname PY@tok@m\endcsname{\def\PY@tc##1{\textcolor[rgb]{0.40,0.40,0.40}{##1}}}
\expandafter\def\csname PY@tok@mh\endcsname{\def\PY@tc##1{\textcolor[rgb]{0.40,0.40,0.40}{##1}}}
\expandafter\def\csname PY@tok@go\endcsname{\def\PY@tc##1{\textcolor[rgb]{0.53,0.53,0.53}{##1}}}
\expandafter\def\csname PY@tok@ge\endcsname{\let\PY@it=\textit}
\expandafter\def\csname PY@tok@vc\endcsname{\def\PY@tc##1{\textcolor[rgb]{0.10,0.09,0.49}{##1}}}
\expandafter\def\csname PY@tok@il\endcsname{\def\PY@tc##1{\textcolor[rgb]{0.40,0.40,0.40}{##1}}}
\expandafter\def\csname PY@tok@cs\endcsname{\let\PY@it=\textit\def\PY@tc##1{\textcolor[rgb]{0.25,0.50,0.50}{##1}}}
\expandafter\def\csname PY@tok@cp\endcsname{\def\PY@tc##1{\textcolor[rgb]{0.74,0.48,0.00}{##1}}}
\expandafter\def\csname PY@tok@gi\endcsname{\def\PY@tc##1{\textcolor[rgb]{0.00,0.63,0.00}{##1}}}
\expandafter\def\csname PY@tok@gh\endcsname{\let\PY@bf=\textbf\def\PY@tc##1{\textcolor[rgb]{0.00,0.00,0.50}{##1}}}
\expandafter\def\csname PY@tok@ni\endcsname{\let\PY@bf=\textbf\def\PY@tc##1{\textcolor[rgb]{0.60,0.60,0.60}{##1}}}
\expandafter\def\csname PY@tok@nl\endcsname{\def\PY@tc##1{\textcolor[rgb]{0.63,0.63,0.00}{##1}}}
\expandafter\def\csname PY@tok@nn\endcsname{\let\PY@bf=\textbf\def\PY@tc##1{\textcolor[rgb]{0.00,0.00,1.00}{##1}}}
\expandafter\def\csname PY@tok@no\endcsname{\def\PY@tc##1{\textcolor[rgb]{0.53,0.00,0.00}{##1}}}
\expandafter\def\csname PY@tok@na\endcsname{\def\PY@tc##1{\textcolor[rgb]{0.49,0.56,0.16}{##1}}}
\expandafter\def\csname PY@tok@nb\endcsname{\def\PY@tc##1{\textcolor[rgb]{0.00,0.50,0.00}{##1}}}
\expandafter\def\csname PY@tok@nc\endcsname{\let\PY@bf=\textbf\def\PY@tc##1{\textcolor[rgb]{0.00,0.00,1.00}{##1}}}
\expandafter\def\csname PY@tok@nd\endcsname{\def\PY@tc##1{\textcolor[rgb]{0.67,0.13,1.00}{##1}}}
\expandafter\def\csname PY@tok@ne\endcsname{\let\PY@bf=\textbf\def\PY@tc##1{\textcolor[rgb]{0.82,0.25,0.23}{##1}}}
\expandafter\def\csname PY@tok@nf\endcsname{\def\PY@tc##1{\textcolor[rgb]{0.00,0.00,1.00}{##1}}}
\expandafter\def\csname PY@tok@si\endcsname{\let\PY@bf=\textbf\def\PY@tc##1{\textcolor[rgb]{0.73,0.40,0.53}{##1}}}
\expandafter\def\csname PY@tok@s2\endcsname{\def\PY@tc##1{\textcolor[rgb]{0.73,0.13,0.13}{##1}}}
\expandafter\def\csname PY@tok@vi\endcsname{\def\PY@tc##1{\textcolor[rgb]{0.10,0.09,0.49}{##1}}}
\expandafter\def\csname PY@tok@nt\endcsname{\let\PY@bf=\textbf\def\PY@tc##1{\textcolor[rgb]{0.00,0.50,0.00}{##1}}}
\expandafter\def\csname PY@tok@nv\endcsname{\def\PY@tc##1{\textcolor[rgb]{0.10,0.09,0.49}{##1}}}
\expandafter\def\csname PY@tok@s1\endcsname{\def\PY@tc##1{\textcolor[rgb]{0.73,0.13,0.13}{##1}}}
\expandafter\def\csname PY@tok@kd\endcsname{\let\PY@bf=\textbf\def\PY@tc##1{\textcolor[rgb]{0.00,0.50,0.00}{##1}}}
\expandafter\def\csname PY@tok@sh\endcsname{\def\PY@tc##1{\textcolor[rgb]{0.73,0.13,0.13}{##1}}}
\expandafter\def\csname PY@tok@sc\endcsname{\def\PY@tc##1{\textcolor[rgb]{0.73,0.13,0.13}{##1}}}
\expandafter\def\csname PY@tok@sx\endcsname{\def\PY@tc##1{\textcolor[rgb]{0.00,0.50,0.00}{##1}}}
\expandafter\def\csname PY@tok@bp\endcsname{\def\PY@tc##1{\textcolor[rgb]{0.00,0.50,0.00}{##1}}}
\expandafter\def\csname PY@tok@c1\endcsname{\let\PY@it=\textit\def\PY@tc##1{\textcolor[rgb]{0.25,0.50,0.50}{##1}}}
\expandafter\def\csname PY@tok@kc\endcsname{\let\PY@bf=\textbf\def\PY@tc##1{\textcolor[rgb]{0.00,0.50,0.00}{##1}}}
\expandafter\def\csname PY@tok@c\endcsname{\let\PY@it=\textit\def\PY@tc##1{\textcolor[rgb]{0.25,0.50,0.50}{##1}}}
\expandafter\def\csname PY@tok@mf\endcsname{\def\PY@tc##1{\textcolor[rgb]{0.40,0.40,0.40}{##1}}}
\expandafter\def\csname PY@tok@err\endcsname{\def\PY@bc##1{\setlength{\fboxsep}{0pt}\fcolorbox[rgb]{1.00,0.00,0.00}{1,1,1}{\strut ##1}}}
\expandafter\def\csname PY@tok@mb\endcsname{\def\PY@tc##1{\textcolor[rgb]{0.40,0.40,0.40}{##1}}}
\expandafter\def\csname PY@tok@ss\endcsname{\def\PY@tc##1{\textcolor[rgb]{0.10,0.09,0.49}{##1}}}
\expandafter\def\csname PY@tok@sr\endcsname{\def\PY@tc##1{\textcolor[rgb]{0.73,0.40,0.53}{##1}}}
\expandafter\def\csname PY@tok@mo\endcsname{\def\PY@tc##1{\textcolor[rgb]{0.40,0.40,0.40}{##1}}}
\expandafter\def\csname PY@tok@kn\endcsname{\let\PY@bf=\textbf\def\PY@tc##1{\textcolor[rgb]{0.00,0.50,0.00}{##1}}}
\expandafter\def\csname PY@tok@mi\endcsname{\def\PY@tc##1{\textcolor[rgb]{0.40,0.40,0.40}{##1}}}
\expandafter\def\csname PY@tok@gp\endcsname{\let\PY@bf=\textbf\def\PY@tc##1{\textcolor[rgb]{0.00,0.00,0.50}{##1}}}
\expandafter\def\csname PY@tok@o\endcsname{\def\PY@tc##1{\textcolor[rgb]{0.40,0.40,0.40}{##1}}}
\expandafter\def\csname PY@tok@kr\endcsname{\let\PY@bf=\textbf\def\PY@tc##1{\textcolor[rgb]{0.00,0.50,0.00}{##1}}}
\expandafter\def\csname PY@tok@s\endcsname{\def\PY@tc##1{\textcolor[rgb]{0.73,0.13,0.13}{##1}}}
\expandafter\def\csname PY@tok@kp\endcsname{\def\PY@tc##1{\textcolor[rgb]{0.00,0.50,0.00}{##1}}}
\expandafter\def\csname PY@tok@w\endcsname{\def\PY@tc##1{\textcolor[rgb]{0.73,0.73,0.73}{##1}}}
\expandafter\def\csname PY@tok@kt\endcsname{\def\PY@tc##1{\textcolor[rgb]{0.69,0.00,0.25}{##1}}}
\expandafter\def\csname PY@tok@ow\endcsname{\let\PY@bf=\textbf\def\PY@tc##1{\textcolor[rgb]{0.67,0.13,1.00}{##1}}}
\expandafter\def\csname PY@tok@sb\endcsname{\def\PY@tc##1{\textcolor[rgb]{0.73,0.13,0.13}{##1}}}
\expandafter\def\csname PY@tok@k\endcsname{\let\PY@bf=\textbf\def\PY@tc##1{\textcolor[rgb]{0.00,0.50,0.00}{##1}}}
\expandafter\def\csname PY@tok@se\endcsname{\let\PY@bf=\textbf\def\PY@tc##1{\textcolor[rgb]{0.73,0.40,0.13}{##1}}}
\expandafter\def\csname PY@tok@sd\endcsname{\let\PY@it=\textit\def\PY@tc##1{\textcolor[rgb]{0.73,0.13,0.13}{##1}}}

\def\PYZbs{\char`\\}
\def\PYZus{\char`\_}
\def\PYZob{\char`\{}
\def\PYZcb{\char`\}}
\def\PYZca{\char`\^}
\def\PYZam{\char`\&}
\def\PYZlt{\char`\<}
\def\PYZgt{\char`\>}
\def\PYZsh{\char`\#}
\def\PYZpc{\char`\%}
\def\PYZdl{\char`\$}
\def\PYZhy{\char`\-}
\def\PYZsq{\char`\'}
\def\PYZdq{\char`\"}
\def\PYZti{\char`\~}
% for compatibility with earlier versions
\def\PYZat{@}
\def\PYZlb{[}
\def\PYZrb{]}
\makeatother


    % Exact colors from NB
    \definecolor{incolor}{rgb}{0.0, 0.0, 0.5}
    \definecolor{outcolor}{rgb}{0.545, 0.0, 0.0}



    
    % Prevent overflowing lines due to hard-to-break entities
    \sloppy 
    % Setup hyperref package
    \hypersetup{
      breaklinks=true,  % so long urls are correctly broken across lines
      colorlinks=true,
      urlcolor=blue,
      linkcolor=darkorange,
      citecolor=darkgreen,
      }
    % Slightly bigger margins than the latex defaults
    
    \geometry{verbose,tmargin=1in,bmargin=1in,lmargin=1in,rmargin=1in}
    
    

    \begin{document}
    
    
    \maketitle
    
    

    
    \subsection{Task 2}\label{task-2}

\paragraph{Read in data}\label{read-in-data}

    \begin{Verbatim}[commandchars=\\\{\}]
{\color{incolor}In [{\color{incolor}61}]:} \PY{n}{img\PYZus{}pts} \PY{o}{=} \PY{n}{np}\PY{o}{.}\PY{n}{genfromtxt}\PY{p}{(}\PY{l+s}{\PYZdq{}}\PY{l+s}{./Files/image.txt}\PY{l+s}{\PYZdq{}}\PY{p}{)}\PY{o}{.}\PY{n}{T}
         \PY{n}{world\PYZus{}pts} \PY{o}{=} \PY{n}{np}\PY{o}{.}\PY{n}{genfromtxt}\PY{p}{(}\PY{l+s}{\PYZdq{}}\PY{l+s}{./Files/world.txt}\PY{l+s}{\PYZdq{}}\PY{p}{)}\PY{o}{.}\PY{n}{T}
         
         \PY{k}{print} \PY{l+s}{\PYZdq{}}\PY{l+s}{Sample image points:}\PY{l+s+se}{\PYZbs{}n}\PY{l+s}{\PYZdq{}}\PY{p}{,}\PY{n}{img\PYZus{}pts}\PY{p}{[}\PY{l+m+mi}{0}\PY{p}{:}\PY{l+m+mi}{2}\PY{p}{]}\PY{p}{,} \PY{l+s}{\PYZdq{}}\PY{l+s+se}{\PYZbs{}n}\PY{l+s+se}{\PYZbs{}n}\PY{l+s}{Sample world points:}\PY{l+s+se}{\PYZbs{}n}\PY{l+s}{\PYZdq{}}\PY{p}{,} \PY{n}{world\PYZus{}pts}\PY{p}{[}\PY{l+m+mi}{0}\PY{p}{:}\PY{l+m+mi}{2}\PY{p}{]}
\end{Verbatim}

    \begin{Verbatim}[commandchars=\\\{\}]
Sample image points:
[[ 5.11770701  4.76538441]
 [ 5.5236545   3.87032917]] 

Sample world points:
[[ 0.8518447   0.75947939  0.94975928]
 [ 0.55793851  0.01423302  0.59617708]]
    \end{Verbatim}

    \paragraph{Solve for camera matrix P}\label{solve-for-camera-matrix-p}

    \begin{Verbatim}[commandchars=\\\{\}]
{\color{incolor}In [{\color{incolor} }]:} \PY{c}{\PYZsh{} gets row of the form}
        \PY{c}{\PYZsh{} [[X1 Y1 Z1 1 0 0 0 0 −x1X1 −x1Y1 −x1Z1 −x1],}
        \PY{c}{\PYZsh{} [0 0 0 0 X1 Y1 Z1 1 −y1X1 −y1Y1 −y1Z1 −y1]]}
        \PY{k}{def} \PY{n+nf}{get\PYZus{}row}\PY{p}{(}\PY{n}{x\PYZus{}i}\PY{p}{,} \PY{n}{X\PYZus{}w}\PY{p}{)}\PY{p}{:}
            \PY{n}{row} \PY{o}{=} \PY{p}{[}\PY{p}{]}
            \PY{n}{row}\PY{o}{.}\PY{n}{append}\PY{p}{(}\PY{p}{[}\PY{n}{X\PYZus{}w}\PY{p}{[}\PY{l+m+mi}{0}\PY{p}{]}\PY{p}{,}\PY{n}{X\PYZus{}w}\PY{p}{[}\PY{l+m+mi}{1}\PY{p}{]}\PY{p}{,}\PY{n}{X\PYZus{}w}\PY{p}{[}\PY{l+m+mi}{2}\PY{p}{]}\PY{p}{,}\PY{l+m+mi}{1}\PY{p}{,}\PY{l+m+mi}{0}\PY{p}{,}\PY{l+m+mi}{0}\PY{p}{,}\PY{l+m+mi}{0}\PY{p}{,}\PY{l+m+mi}{0}\PY{p}{,}\PY{o}{\PYZhy{}}\PY{n}{x\PYZus{}i}\PY{p}{[}\PY{l+m+mi}{0}\PY{p}{]}\PY{o}{*}\PY{n}{X\PYZus{}w}\PY{p}{[}\PY{l+m+mi}{0}\PY{p}{]}\PY{p}{,}\PY{o}{\PYZhy{}}\PY{n}{x\PYZus{}i}\PY{p}{[}\PY{l+m+mi}{0}\PY{p}{]}\PY{o}{*}\PY{n}{X\PYZus{}w}\PY{p}{[}\PY{l+m+mi}{1}\PY{p}{]}\PY{p}{,}\PY{o}{\PYZhy{}}\PY{n}{x\PYZus{}i}\PY{p}{[}\PY{l+m+mi}{0}\PY{p}{]}\PY{o}{*}\PY{n}{X\PYZus{}w}\PY{p}{[}\PY{l+m+mi}{2}\PY{p}{]}\PY{p}{,}\PY{o}{\PYZhy{}}\PY{n}{x\PYZus{}i}\PY{p}{[}\PY{l+m+mi}{0}\PY{p}{]}\PY{p}{]}\PY{p}{)}
            \PY{n}{row}\PY{o}{.}\PY{n}{append}\PY{p}{(}\PY{p}{[}\PY{l+m+mi}{0}\PY{p}{,}\PY{l+m+mi}{0}\PY{p}{,}\PY{l+m+mi}{0}\PY{p}{,}\PY{l+m+mi}{0}\PY{p}{,}\PY{n}{X\PYZus{}w}\PY{p}{[}\PY{l+m+mi}{0}\PY{p}{]}\PY{p}{,}\PY{n}{X\PYZus{}w}\PY{p}{[}\PY{l+m+mi}{1}\PY{p}{]}\PY{p}{,}\PY{n}{X\PYZus{}w}\PY{p}{[}\PY{l+m+mi}{2}\PY{p}{]}\PY{p}{,}\PY{l+m+mi}{1}\PY{p}{,}\PY{o}{\PYZhy{}}\PY{n}{x\PYZus{}i}\PY{p}{[}\PY{l+m+mi}{1}\PY{p}{]}\PY{o}{*}\PY{n}{X\PYZus{}w}\PY{p}{[}\PY{l+m+mi}{0}\PY{p}{]}\PY{p}{,}\PY{o}{\PYZhy{}}\PY{n}{x\PYZus{}i}\PY{p}{[}\PY{l+m+mi}{1}\PY{p}{]}\PY{o}{*}\PY{n}{X\PYZus{}w}\PY{p}{[}\PY{l+m+mi}{1}\PY{p}{]}\PY{p}{,}\PY{o}{\PYZhy{}}\PY{n}{x\PYZus{}i}\PY{p}{[}\PY{l+m+mi}{1}\PY{p}{]}\PY{o}{*}\PY{n}{X\PYZus{}w}\PY{p}{[}\PY{l+m+mi}{2}\PY{p}{]}\PY{p}{,}\PY{o}{\PYZhy{}}\PY{n}{x\PYZus{}i}\PY{p}{[}\PY{l+m+mi}{1}\PY{p}{]}\PY{p}{]}\PY{p}{)}
            \PY{k}{return} \PY{n}{np}\PY{o}{.}\PY{n}{array}\PY{p}{(}\PY{n}{row}\PY{p}{)}
\end{Verbatim}

    \begin{Verbatim}[commandchars=\\\{\}]
{\color{incolor}In [{\color{incolor}62}]:} \PY{n}{A\PYZus{}mat} \PY{o}{=} \PY{p}{[}\PY{p}{]}
         
         \PY{k}{for} \PY{n}{i} \PY{o+ow}{in} \PY{n}{np}\PY{o}{.}\PY{n}{arange}\PY{p}{(}\PY{l+m+mi}{10}\PY{p}{)}\PY{p}{:}
             \PY{n}{r} \PY{o}{=} \PY{n}{get\PYZus{}row}\PY{p}{(}\PY{n}{img\PYZus{}pts}\PY{p}{[}\PY{n}{i}\PY{p}{]}\PY{p}{,} \PY{n}{world\PYZus{}pts}\PY{p}{[}\PY{n}{i}\PY{p}{]}\PY{p}{)}
             \PY{n}{A\PYZus{}mat}\PY{o}{.}\PY{n}{append}\PY{p}{(}\PY{n}{r}\PY{p}{[}\PY{l+m+mi}{0}\PY{p}{]}\PY{p}{)}
             \PY{n}{A\PYZus{}mat}\PY{o}{.}\PY{n}{append}\PY{p}{(}\PY{n}{r}\PY{p}{[}\PY{l+m+mi}{1}\PY{p}{]}\PY{p}{)}
             
         \PY{n}{A\PYZus{}mat} \PY{o}{=} \PY{n}{np}\PY{o}{.}\PY{n}{array}\PY{p}{(}\PY{n}{A\PYZus{}mat}\PY{p}{)}
         \PY{c}{\PYZsh{} print A\PYZus{}mat.shape}
         
         \PY{n}{V} \PY{o}{=} \PY{n}{np}\PY{o}{.}\PY{n}{linalg}\PY{o}{.}\PY{n}{svd}\PY{p}{(}\PY{n}{A\PYZus{}mat}\PY{p}{)}\PY{p}{[}\PY{l+m+mi}{2}\PY{p}{]}
         \PY{n}{P\PYZus{}mat} \PY{o}{=} \PY{n}{V}\PY{p}{[}\PY{n+nb}{len}\PY{p}{(}\PY{n}{V}\PY{p}{)}\PY{o}{\PYZhy{}}\PY{l+m+mi}{1}\PY{p}{]}\PY{o}{.}\PY{n}{reshape}\PY{p}{(}\PY{l+m+mi}{3}\PY{p}{,}\PY{l+m+mi}{4}\PY{p}{)}
         \PY{c}{\PYZsh{} print P\PYZus{}mat.shape}
         
         \PY{k}{print} \PY{l+s}{\PYZdq{}}\PY{l+s}{We get the camera matrix P as:}\PY{l+s+se}{\PYZbs{}n}\PY{l+s}{\PYZdq{}}\PY{p}{,}\PY{n}{P\PYZus{}mat}
\end{Verbatim}

    \begin{Verbatim}[commandchars=\\\{\}]
We get the camera matrix P as:
[[  1.27000127e-01   2.54000254e-01   3.81000381e-01   5.08000508e-01]
 [  5.08000508e-01   3.81000381e-01   2.54000254e-01   1.27000127e-01]
 [  1.27000127e-01   1.94289029e-16   1.27000127e-01  -2.22044605e-16]]
    \end{Verbatim}

    \paragraph{Verify answer by reprojecting world
points}\label{verify-answer-by-reprojecting-world-points}

As can be seen, the image points and the reprojected image points
closely correspond to each other. The difference between the 2 are
almost 0

    \begin{Verbatim}[commandchars=\\\{\}]
{\color{incolor}In [{\color{incolor}64}]:} \PY{n}{img\PYZus{}points\PYZus{}reprojected} \PY{o}{=} \PY{p}{[}\PY{p}{]}
         \PY{k}{for} \PY{n}{X} \PY{o+ow}{in} \PY{n}{world\PYZus{}pts}\PY{p}{:}
             \PY{n}{img\PYZus{}homog} \PY{o}{=} \PY{n}{P\PYZus{}mat}\PY{o}{.}\PY{n}{dot}\PY{p}{(}\PY{n}{np}\PY{o}{.}\PY{n}{append}\PY{p}{(}\PY{n}{X}\PY{p}{,}\PY{l+m+mi}{1}\PY{p}{)}\PY{o}{.}\PY{n}{T}\PY{p}{)}
             \PY{n}{img\PYZus{}points\PYZus{}reprojected}\PY{o}{.}\PY{n}{append}\PY{p}{(}\PY{p}{[}\PY{n}{img\PYZus{}homog}\PY{p}{[}\PY{l+m+mi}{0}\PY{p}{]}\PY{o}{/}\PY{n}{img\PYZus{}homog}\PY{p}{[}\PY{l+m+mi}{2}\PY{p}{]}\PY{p}{,} \PY{n}{img\PYZus{}homog}\PY{p}{[}\PY{l+m+mi}{1}\PY{p}{]}\PY{o}{/}\PY{n}{img\PYZus{}homog}\PY{p}{[}\PY{l+m+mi}{2}\PY{p}{]}\PY{p}{]}\PY{p}{)}
         \PY{k}{print} \PY{l+s}{\PYZdq{}}\PY{l+s}{Given image points:}\PY{l+s+se}{\PYZbs{}n}\PY{l+s}{\PYZdq{}}\PY{p}{,}\PY{n}{img\PYZus{}pts}
         \PY{k}{print} \PY{l+s}{\PYZdq{}}\PY{l+s+se}{\PYZbs{}n}\PY{l+s+se}{\PYZbs{}n}\PY{l+s}{Reprojected image points:}\PY{l+s+se}{\PYZbs{}n}\PY{l+s}{\PYZdq{}}\PY{p}{,}\PY{n}{img\PYZus{}points\PYZus{}reprojected}
         \PY{k}{print} \PY{l+s}{\PYZdq{}}\PY{l+s+se}{\PYZbs{}n}\PY{l+s+se}{\PYZbs{}n}\PY{l+s}{Difference:}\PY{l+s+se}{\PYZbs{}n}\PY{l+s}{\PYZdq{}}\PY{p}{,} \PY{n}{img\PYZus{}pts} \PY{o}{\PYZhy{}} \PY{n}{img\PYZus{}points\PYZus{}reprojected}
\end{Verbatim}

    \begin{Verbatim}[commandchars=\\\{\}]
Given image points:
[[  5.11770701   4.76538441]
 [  5.5236545    3.87032917]
 [  7.16310171   7.35942066]
 [  5.22216628   4.4279585 ]
 [  5.60479614   4.67483648]
 [ 13.59494885  10.05215495]
 [  8.73452189   5.56420531]
 [  6.22433952   3.90821885]
 [  9.74763886   6.90423723]
 [  5.09031079   4.5508513 ]]


Reprojected image points:
[[5.1177070095845512, 4.7653844064445083], [5.5236545022530548, 3.8703291749533317], [7.1631017123476957, 7.3594206563832989], [5.2221662777362265, 4.4279585004935562], [5.6047961393951118, 4.6748364822514068], [13.594948846275605, 10.052154948111069], [8.7345218880620887, 5.564205313535151], [6.224339521255974, 3.9082188503387831], [9.7476388596450469, 6.9042372297085111], [5.0903107941755881, 4.5508513049774946]]


Difference:
[[ -2.66453526e-15  -2.66453526e-15]
 [ -4.44089210e-15  -4.44089210e-16]
 [  0.00000000e+00   1.77635684e-15]
 [ -1.77635684e-15   0.00000000e+00]
 [ -8.88178420e-16   1.77635684e-15]
 [  0.00000000e+00   0.00000000e+00]
 [ -3.55271368e-15  -5.32907052e-15]
 [ -5.32907052e-15  -2.66453526e-15]
 [ -1.77635684e-15  -2.66453526e-15]
 [ -2.66453526e-15   0.00000000e+00]]
    \end{Verbatim}

    \paragraph{World coordinates of the projection center of the
camera}\label{world-coordinates-of-the-projection-center-of-the-camera}

    \begin{Verbatim}[commandchars=\\\{\}]
{\color{incolor}In [{\color{incolor}65}]:} \PY{n}{V\PYZus{}p} \PY{o}{=} \PY{n}{np}\PY{o}{.}\PY{n}{linalg}\PY{o}{.}\PY{n}{svd}\PY{p}{(}\PY{n}{P\PYZus{}mat}\PY{p}{)}\PY{p}{[}\PY{l+m+mi}{2}\PY{p}{]}
         \PY{n}{camera\PYZus{}vec} \PY{o}{=} \PY{n}{V\PYZus{}p}\PY{p}{[}\PY{n+nb}{len}\PY{p}{(}\PY{n}{V\PYZus{}p}\PY{p}{)}\PY{o}{\PYZhy{}}\PY{l+m+mi}{1}\PY{p}{]}
         
         \PY{n}{camera\PYZus{}vec} \PY{o}{=} \PY{n}{camera\PYZus{}vec}\PY{p}{[}\PY{l+m+mi}{0}\PY{p}{:}\PY{l+m+mi}{3}\PY{p}{]}\PY{o}{/}\PY{n}{camera\PYZus{}vec}\PY{p}{[}\PY{l+m+mi}{3}\PY{p}{]}
         \PY{k}{print} \PY{l+s}{\PYZdq{}}\PY{l+s}{World coordinates of projection center of camera:}\PY{l+s+se}{\PYZbs{}n}\PY{l+s}{\PYZdq{}}\PY{p}{,}\PY{n}{camera\PYZus{}vec}
\end{Verbatim}

    \begin{Verbatim}[commandchars=\\\{\}]
World coordinates of projection center of camera:
[ 1. -1. -1.]
    \end{Verbatim}

    \subsection{Task 4}\label{task-4}

\subsubsection{Question 1}\label{question-1}

\paragraph{a. What is the relation between the epipole and the epipolar
line?}\label{a.-what-is-the-relation-between-the-epipole-and-the-epipolar-line}

All epipolar lines in image 1 will intersect at the epipole
corresponding to image 1

\paragraph{b. How many epipoles exist in a set of two cameras?
Explain.}\label{b.-how-many-epipoles-exist-in-a-set-of-two-cameras-explain.}

When there are 2 cameras, the epipoles will be the point of intersection
of the line joining the camera centres ie. the baseline and the image
planes. Since there are 2 image planes, we can expect there are 2
epipoles

\paragraph{c. Given the fundamental matrix F in a set of two cameras,
express the epipoles in the set as a function of
F}\label{c.-given-the-fundamental-matrix-f-in-a-set-of-two-cameras-express-the-epipoles-in-the-set-as-a-function-of-f}

Let the epipoles of first and second image be e and e'

We have the property x'TFx = 0

For any point x (other than e) the epipolar line l' = Fx contains the
epipole e'.

Thus e' satisfies e'T(Fx)=(e'TF)x = 0 for all x.

Thus, e'TF = 0

Using a similar argument, we can obtain that Fe = 0

i.e.~e' and e are the left and right null-space of F respectively

e' = null(FT)\\
e = null(F)

\paragraph{d. Second camera translated along Y-axis by distance
d}\label{d.-second-camera-translated-along-y-axis-by-distance-d}

The epipole is the point of intersection of the baseline with the image
plane. In the setup below, it is seen that the 2 image planes are
parallel to each other and the baseline. So, the image planes and the
baseline will intersect each other at positive and negative infinity and
the epipolar lines will be parallel to the baseline.

    \begin{Verbatim}[commandchars=\\\{\}]
{\color{incolor}In [{\color{incolor}4}]:} \PY{k}{def} \PY{n+nf}{disp}\PY{p}{(}\PY{n}{img}\PY{p}{,} \PY{n}{h}\PY{o}{=}\PY{l+m+mi}{10}\PY{p}{,} \PY{n}{w}\PY{o}{=}\PY{l+m+mi}{20}\PY{p}{,} \PY{n}{title}\PY{o}{=}\PY{p}{[}\PY{p}{]}\PY{p}{)}\PY{p}{:}
            \PY{n}{plt}\PY{o}{.}\PY{n}{figure}\PY{p}{(}\PY{n}{figsize}\PY{o}{=}\PY{p}{(}\PY{n}{w}\PY{p}{,}\PY{n}{h}\PY{p}{)}\PY{p}{)}
            \PY{k}{if}\PY{p}{(}\PY{n+nb}{len}\PY{p}{(}\PY{n}{title}\PY{p}{)}\PY{o}{\PYZgt{}}\PY{l+m+mi}{0}\PY{p}{)}\PY{p}{:}
                \PY{n}{plt}\PY{o}{.}\PY{n}{title}\PY{p}{(}\PY{n}{title}\PY{p}{)}
            \PY{n}{plt}\PY{o}{.}\PY{n}{imshow}\PY{p}{(}\PY{n}{img}\PY{p}{,} \PY{n}{cmap}\PY{o}{=}\PY{n}{cm}\PY{o}{.}\PY{n}{gray}\PY{p}{)}
\end{Verbatim}

    \begin{Verbatim}[commandchars=\\\{\}]
{\color{incolor}In [{\color{incolor}18}]:} \PY{n}{disp}\PY{p}{(}\PY{n}{cv2}\PY{o}{.}\PY{n}{imread}\PY{p}{(}\PY{l+s}{\PYZsq{}}\PY{l+s}{./part\PYZus{}d.png}\PY{l+s}{\PYZsq{}}\PY{p}{)}\PY{p}{)}
\end{Verbatim}

    \begin{center}
    \adjustimage{max size={0.9\linewidth}{0.9\paperheight}}{HW4_files/HW4_11_0.png}
    \end{center}
    { \hspace*{\fill} \\}
    
    \paragraph{e. Second camera rotated with a rotation matrix
R}\label{e.-second-camera-rotated-with-a-rotation-matrix-r}

Reference:\\
http://www1.cs.columbia.edu/\textasciitilde{}jebara/htmlpapers/SFM/node8.html
http://www.ee.oulu.fi/research/imag/courses/Sturm/moons09.pdf

When the second camera is only rotated with respect to the first, it
means there will be no single line connecting the 2 camera centers, and
hence, \textbf{there will be no baseline}. without a baseline, we cannot
determine points which intersect the image planes. So, \textbf{the
epipoles will be undefined} for this case. So, we cannot define a
fundamental matrix or epipolar lines for this system.

Also (ref http://www.ee.oulu.fi/research/imag/courses/Sturm/moons09.pdf,
pg 383), given the intrinsic matrix is identity, we will be able to
obtain a homography matrix transformation (with only rotation component)
between to map a scene point M's image points m1 and m2

ie. m2 = R.m1

    \begin{Verbatim}[commandchars=\\\{\}]
{\color{incolor}In [{\color{incolor}17}]:} \PY{n}{disp}\PY{p}{(}\PY{n}{cv2}\PY{o}{.}\PY{n}{imread}\PY{p}{(}\PY{l+s}{\PYZsq{}}\PY{l+s}{./part\PYZus{}e.png}\PY{l+s}{\PYZsq{}}\PY{p}{)}\PY{p}{,}\PY{l+m+mi}{5}\PY{p}{,}\PY{l+m+mi}{10}\PY{p}{)}
\end{Verbatim}

    \begin{center}
    \adjustimage{max size={0.9\linewidth}{0.9\paperheight}}{HW4_files/HW4_13_0.png}
    \end{center}
    { \hspace*{\fill} \\}
    
    \subsubsection{Question 3}\label{question-3}

\paragraph{a. Find matching interest points using SIFT or
Harris}\label{a.-find-matching-interest-points-using-sift-or-harris}

First we read in the images

    \begin{Verbatim}[commandchars=\\\{\}]
{\color{incolor}In [{\color{incolor}13}]:} \PY{c}{\PYZsh{} first we read in the images}
         \PY{n}{img1} \PY{o}{=} \PY{n}{cv2}\PY{o}{.}\PY{n}{imread}\PY{p}{(}\PY{l+s}{\PYZsq{}}\PY{l+s}{./Files/chapel2.png}\PY{l+s}{\PYZsq{}}\PY{p}{,}\PY{l+m+mi}{0}\PY{p}{)}
         \PY{n}{img2} \PY{o}{=} \PY{n}{cv2}\PY{o}{.}\PY{n}{imread}\PY{p}{(}\PY{l+s}{\PYZsq{}}\PY{l+s}{./Files/chapel1.png}\PY{l+s}{\PYZsq{}}\PY{p}{,}\PY{l+m+mi}{0}\PY{p}{)}
         \PY{n}{disp}\PY{p}{(}\PY{n}{img1}\PY{p}{,}\PY{l+m+mi}{3}\PY{p}{,}\PY{l+m+mi}{6}\PY{p}{,}\PY{l+s}{\PYZdq{}}\PY{l+s}{Chapel Left}\PY{l+s}{\PYZdq{}}\PY{p}{)}
         \PY{n}{disp}\PY{p}{(}\PY{n}{img2}\PY{p}{,}\PY{l+m+mi}{3}\PY{p}{,}\PY{l+m+mi}{6}\PY{p}{,}\PY{l+s}{\PYZdq{}}\PY{l+s}{Chapel Right}\PY{l+s}{\PYZdq{}}\PY{p}{)}
\end{Verbatim}

    \begin{center}
    \adjustimage{max size={0.9\linewidth}{0.9\paperheight}}{HW4_files/HW4_15_0.png}
    \end{center}
    { \hspace*{\fill} \\}
    
    \begin{center}
    \adjustimage{max size={0.9\linewidth}{0.9\paperheight}}{HW4_files/HW4_15_1.png}
    \end{center}
    { \hspace*{\fill} \\}
    
    We use the SIFT Detector to identify keypoints in the left and right
images

    \begin{Verbatim}[commandchars=\\\{\}]
{\color{incolor}In [{\color{incolor}14}]:} \PY{n}{sift} \PY{o}{=} \PY{n}{cv2}\PY{o}{.}\PY{n}{SIFT}\PY{p}{(}\PY{p}{)}
         
         \PY{n}{kp1}\PY{p}{,} \PY{n}{des1} \PY{o}{=} \PY{n}{sift}\PY{o}{.}\PY{n}{detectAndCompute}\PY{p}{(}\PY{n}{img1}\PY{p}{,}\PY{n+nb+bp}{None}\PY{p}{)}
         \PY{n}{img1c} \PY{o}{=} \PY{n}{cv2}\PY{o}{.}\PY{n}{drawKeypoints}\PY{p}{(}\PY{n}{img1}\PY{p}{,}\PY{n}{kp1}\PY{p}{)}
         \PY{n}{disp}\PY{p}{(}\PY{n}{img1c}\PY{p}{,}\PY{l+m+mi}{3}\PY{p}{,}\PY{l+m+mi}{6}\PY{p}{,}\PY{l+s}{\PYZdq{}}\PY{l+s}{Chapel Left With SIFT Points}\PY{l+s}{\PYZdq{}}\PY{p}{)}
         
         \PY{n}{kp2}\PY{p}{,} \PY{n}{des2} \PY{o}{=} \PY{n}{sift}\PY{o}{.}\PY{n}{detectAndCompute}\PY{p}{(}\PY{n}{img2}\PY{p}{,}\PY{n+nb+bp}{None}\PY{p}{)}
         \PY{n}{img2c} \PY{o}{=} \PY{n}{cv2}\PY{o}{.}\PY{n}{drawKeypoints}\PY{p}{(}\PY{n}{img2}\PY{p}{,}\PY{n}{kp2}\PY{p}{)}
         \PY{n}{disp}\PY{p}{(}\PY{n}{img2c}\PY{p}{,}\PY{l+m+mi}{3}\PY{p}{,}\PY{l+m+mi}{6}\PY{p}{,}\PY{l+s}{\PYZdq{}}\PY{l+s}{Chapel Right With SIFT Points}\PY{l+s}{\PYZdq{}}\PY{p}{)}
\end{Verbatim}

    \begin{center}
    \adjustimage{max size={0.9\linewidth}{0.9\paperheight}}{HW4_files/HW4_17_0.png}
    \end{center}
    { \hspace*{\fill} \\}
    
    \begin{center}
    \adjustimage{max size={0.9\linewidth}{0.9\paperheight}}{HW4_files/HW4_17_1.png}
    \end{center}
    { \hspace*{\fill} \\}
    
    We use the Brute Force Matcher to match the keypoints between the left
and right images. We get a total of 461 matches

    \begin{Verbatim}[commandchars=\\\{\}]
{\color{incolor}In [{\color{incolor}15}]:} \PY{c}{\PYZsh{} we match using BFMatcher}
         \PY{n}{bf} \PY{o}{=} \PY{n}{cv2}\PY{o}{.}\PY{n}{BFMatcher}\PY{p}{(}\PY{n}{cv2}\PY{o}{.}\PY{n}{NORM\PYZus{}L2}\PY{p}{,} \PY{n}{crossCheck}\PY{o}{=}\PY{n+nb+bp}{True}\PY{p}{)}
         
         \PY{c}{\PYZsh{} Match descriptors.}
         \PY{n}{matches} \PY{o}{=} \PY{n}{bf}\PY{o}{.}\PY{n}{match}\PY{p}{(}\PY{n}{des1}\PY{p}{,}\PY{n}{des2}\PY{p}{)}
         \PY{n}{matches} \PY{o}{=} \PY{n+nb}{sorted}\PY{p}{(}\PY{n}{matches}\PY{p}{,} \PY{n}{key} \PY{o}{=} \PY{k}{lambda} \PY{n}{x}\PY{p}{:}\PY{n}{x}\PY{o}{.}\PY{n}{distance}\PY{p}{)}
         
         \PY{k}{print} \PY{l+s}{\PYZdq{}}\PY{l+s}{Number of matches:}\PY{l+s}{\PYZdq{}}\PY{p}{,} \PY{n+nb}{len}\PY{p}{(}\PY{n}{matches}\PY{p}{)}
\end{Verbatim}

    \begin{Verbatim}[commandchars=\\\{\}]
Number of matches: 463
    \end{Verbatim}

    \paragraph{b. Use the set of matched points you found to estimate the
fundamental matrix F automatically using RANSAC and the normalized
8-point
algorithm.}\label{b.-use-the-set-of-matched-points-you-found-to-estimate-the-fundamental-matrix-f-automatically-using-ransac-and-the-normalized-8-point-algorithm.}

We use OpenCV's findFundamentalMat method to obtain the fundamental
matrix using RANSAC. The underlying algorithm will be the 8-point
algorithm.

\textbf{i. Indicate what test you used for deciding inlier vs.~outlier.}

We use the param1 parameter in the findFundamentalMat method as the
test. It is the maximum distance from a point to an epipolar line in
pixels, beyond which the point is considered an outlier and is not used
for computing the final fundamental matrix. We set it to a value of 3 as
the epipolar lines should pass within around 1 pixel of the inlier
points as per the problem statement. If we reduce it to lower than 3, we
don't get enough inlier points in all the surfaces in the image and this
results in an inaccurate fundamental matrix and consequently, inaccurate
epipolar lines.

\textbf{ii. Display F after normalizing to unit length.}

    \begin{Verbatim}[commandchars=\\\{\}]
{\color{incolor}In [{\color{incolor}16}]:} \PY{n}{pts1} \PY{o}{=} \PY{p}{[}\PY{p}{]}
         \PY{n}{pts2} \PY{o}{=} \PY{p}{[}\PY{p}{]}
         
         \PY{k}{for} \PY{n}{i}\PY{p}{,}\PY{n}{m} \PY{o+ow}{in} \PY{n+nb}{enumerate}\PY{p}{(}\PY{n}{matches}\PY{p}{)}\PY{p}{:}
             \PY{n}{pts2}\PY{o}{.}\PY{n}{append}\PY{p}{(}\PY{n}{kp2}\PY{p}{[}\PY{n}{m}\PY{o}{.}\PY{n}{trainIdx}\PY{p}{]}\PY{o}{.}\PY{n}{pt}\PY{p}{)}
             \PY{n}{pts1}\PY{o}{.}\PY{n}{append}\PY{p}{(}\PY{n}{kp1}\PY{p}{[}\PY{n}{m}\PY{o}{.}\PY{n}{queryIdx}\PY{p}{]}\PY{o}{.}\PY{n}{pt}\PY{p}{)}
         
         \PY{n}{pts1} \PY{o}{=} \PY{n}{np}\PY{o}{.}\PY{n}{float32}\PY{p}{(}\PY{n}{pts1}\PY{p}{)}
         \PY{n}{pts2} \PY{o}{=} \PY{n}{np}\PY{o}{.}\PY{n}{float32}\PY{p}{(}\PY{n}{pts2}\PY{p}{)}
         \PY{n}{F\PYZus{}prenorm}\PY{p}{,} \PY{n}{mask} \PY{o}{=} \PY{n}{cv2}\PY{o}{.}\PY{n}{findFundamentalMat}\PY{p}{(}\PY{n}{pts1}\PY{p}{,}\PY{n}{pts2}\PY{p}{,}\PY{n}{cv2}\PY{o}{.}\PY{n}{FM\PYZus{}RANSAC}\PY{p}{,}\PY{l+m+mi}{3}\PY{p}{)}
         \PY{k}{print} \PY{l+s}{\PYZdq{}}\PY{l+s}{Pre\PYZhy{}Normalized Fundamental Matrix:}\PY{l+s+se}{\PYZbs{}n}\PY{l+s+se}{\PYZbs{}n}\PY{l+s}{\PYZdq{}}\PY{p}{,}\PY{n}{F\PYZus{}prenorm}
         
         \PY{n}{F} \PY{o}{=} \PY{n}{F\PYZus{}prenorm}\PY{o}{/}\PY{n}{np}\PY{o}{.}\PY{n}{linalg}\PY{o}{.}\PY{n}{norm}\PY{p}{(}\PY{n}{F\PYZus{}prenorm}\PY{p}{)}
         \PY{k}{print} \PY{l+s}{\PYZdq{}}\PY{l+s+se}{\PYZbs{}n}\PY{l+s+se}{\PYZbs{}n}\PY{l+s}{Normalized Fundamental Matrix:}\PY{l+s+se}{\PYZbs{}n}\PY{l+s+se}{\PYZbs{}n}\PY{l+s}{\PYZdq{}}\PY{p}{,}\PY{n}{F}
\end{Verbatim}

    \begin{Verbatim}[commandchars=\\\{\}]
Pre-Normalized Fundamental Matrix:

[[ -6.70832358e-07  -2.01041064e-05   9.96680080e-03]
 [  8.09922509e-06   1.06504817e-05   1.28055730e-01]
 [ -7.81529520e-03  -1.26661976e-01   1.00000000e+00]]


Normalized Fundamental Matrix:

[[ -6.60157458e-07  -1.97841914e-05   9.80819993e-03]
 [  7.97034279e-06   1.04810015e-05   1.26017990e-01]
 [ -7.69093106e-03  -1.24646414e-01   9.84087082e-01]]
    \end{Verbatim}

    \textbf{iii. Plot the outliers with green dots on top of the first
image.}

We plot the outliers in green and inliers in red for the first image

    \begin{Verbatim}[commandchars=\\\{\}]
{\color{incolor}In [{\color{incolor}17}]:} \PY{c}{\PYZsh{} We then store only inlier points}
         \PY{n}{pts1} \PY{o}{=} \PY{n}{pts1}\PY{p}{[}\PY{n}{mask}\PY{o}{.}\PY{n}{ravel}\PY{p}{(}\PY{p}{)}\PY{o}{==}\PY{l+m+mi}{1}\PY{p}{]}
         \PY{n}{pts2} \PY{o}{=} \PY{n}{pts2}\PY{p}{[}\PY{n}{mask}\PY{o}{.}\PY{n}{ravel}\PY{p}{(}\PY{p}{)}\PY{o}{==}\PY{l+m+mi}{1}\PY{p}{]}
         
         \PY{c}{\PYZsh{} We also get the outlier points to display}
         \PY{n}{outlier\PYZus{}kp1} \PY{o}{=} \PY{p}{[}\PY{p}{]}
         \PY{n}{inlier\PYZus{}kp1} \PY{o}{=} \PY{p}{[}\PY{p}{]}
         \PY{k}{for} \PY{n}{i}\PY{p}{,} \PY{n}{mv} \PY{o+ow}{in} \PY{n+nb}{enumerate}\PY{p}{(}\PY{n}{mask}\PY{p}{)}\PY{p}{:}
             \PY{n}{match} \PY{o}{=} \PY{n}{matches}\PY{p}{[}\PY{n}{i}\PY{p}{]}
             \PY{k}{if}\PY{p}{(}\PY{n}{mv}\PY{o}{.}\PY{n}{ravel}\PY{p}{(}\PY{p}{)}\PY{o}{==}\PY{l+m+mi}{0}\PY{p}{)}\PY{p}{:}
                 \PY{n}{outlier\PYZus{}kp1}\PY{o}{.}\PY{n}{append}\PY{p}{(}\PY{n}{kp1}\PY{p}{[}\PY{n}{match}\PY{o}{.}\PY{n}{queryIdx}\PY{p}{]}\PY{p}{)}
             \PY{k}{else}\PY{p}{:}
                 \PY{n}{inlier\PYZus{}kp1}\PY{o}{.}\PY{n}{append}\PY{p}{(}\PY{n}{kp1}\PY{p}{[}\PY{n}{match}\PY{o}{.}\PY{n}{queryIdx}\PY{p}{]}\PY{p}{)}
                 
         \PY{n}{img1o} \PY{o}{=} \PY{n}{cv2}\PY{o}{.}\PY{n}{drawKeypoints}\PY{p}{(}\PY{n}{img1}\PY{p}{,}\PY{n}{outlier\PYZus{}kp1}\PY{p}{,}\PY{n}{color}\PY{o}{=}\PY{p}{[}\PY{l+m+mi}{0}\PY{p}{,}\PY{l+m+mi}{255}\PY{p}{,}\PY{l+m+mi}{0}\PY{p}{]}\PY{p}{)}
         \PY{n}{img1i} \PY{o}{=} \PY{n}{cv2}\PY{o}{.}\PY{n}{drawKeypoints}\PY{p}{(}\PY{n}{img1}\PY{p}{,}\PY{n}{inlier\PYZus{}kp1}\PY{p}{,}\PY{n}{color}\PY{o}{=}\PY{p}{[}\PY{l+m+mi}{255}\PY{p}{,}\PY{l+m+mi}{0}\PY{p}{,}\PY{l+m+mi}{0}\PY{p}{]}\PY{p}{)}
         \PY{n}{disp}\PY{p}{(}\PY{n}{img1i}\PY{p}{,}\PY{l+m+mi}{3}\PY{p}{,}\PY{l+m+mi}{6}\PY{p}{,}\PY{l+s}{\PYZdq{}}\PY{l+s}{Inlier Points}\PY{l+s}{\PYZdq{}}\PY{p}{)}
         \PY{n}{disp}\PY{p}{(}\PY{n}{img1o}\PY{p}{,}\PY{l+m+mi}{3}\PY{p}{,}\PY{l+m+mi}{6}\PY{p}{,}\PY{l+s}{\PYZdq{}}\PY{l+s}{Outlier Points}\PY{l+s}{\PYZdq{}}\PY{p}{)}
\end{Verbatim}

    \begin{center}
    \adjustimage{max size={0.9\linewidth}{0.9\paperheight}}{HW4_files/HW4_23_0.png}
    \end{center}
    { \hspace*{\fill} \\}
    
    \begin{center}
    \adjustimage{max size={0.9\linewidth}{0.9\paperheight}}{HW4_files/HW4_23_1.png}
    \end{center}
    { \hspace*{\fill} \\}
    
    \paragraph{c. Choose 7 sets of matching points that are well separated
(can be randomly chosen). Plot the corresponding epipolar lines (in red)
and the points (in green) on each image. Show the two images (with
plotted points and lines) next to each
other.}\label{c.-choose-7-sets-of-matching-points-that-are-well-separated-can-be-randomly-chosen.-plot-the-corresponding-epipolar-lines-in-red-and-the-points-in-green-on-each-image.-show-the-two-images-with-plotted-points-and-lines-next-to-each-other.}

    \begin{Verbatim}[commandchars=\\\{\}]
{\color{incolor}In [{\color{incolor}31}]:} \PY{k}{def} \PY{n+nf}{drawlines}\PY{p}{(}\PY{n}{img\PYZus{}left\PYZus{}orig}\PY{p}{,}\PY{n}{img\PYZus{}right\PYZus{}orig}\PY{p}{,}\PY{n}{lines}\PY{p}{,}\PY{n}{pts1}\PY{p}{,}\PY{n}{pts2}\PY{p}{)}\PY{p}{:}
             \PY{n}{r}\PY{p}{,}\PY{n}{c} \PY{o}{=} \PY{n}{img\PYZus{}left\PYZus{}orig}\PY{o}{.}\PY{n}{shape}
             
             \PY{n}{img\PYZus{}left} \PY{o}{=} \PY{n}{np}\PY{o}{.}\PY{n}{copy}\PY{p}{(}\PY{n}{img\PYZus{}left\PYZus{}orig}\PY{p}{)}
             \PY{n}{img\PYZus{}right} \PY{o}{=} \PY{n}{np}\PY{o}{.}\PY{n}{copy}\PY{p}{(}\PY{n}{img\PYZus{}right\PYZus{}orig}\PY{p}{)}
             
             \PY{n}{img\PYZus{}left} \PY{o}{=} \PY{n}{cv2}\PY{o}{.}\PY{n}{cvtColor}\PY{p}{(}\PY{n}{img\PYZus{}left\PYZus{}orig}\PY{p}{,}\PY{n}{cv2}\PY{o}{.}\PY{n}{COLOR\PYZus{}GRAY2BGR}\PY{p}{)}
             \PY{n}{img\PYZus{}right} \PY{o}{=} \PY{n}{cv2}\PY{o}{.}\PY{n}{cvtColor}\PY{p}{(}\PY{n}{img\PYZus{}right\PYZus{}orig}\PY{p}{,}\PY{n}{cv2}\PY{o}{.}\PY{n}{COLOR\PYZus{}GRAY2BGR}\PY{p}{)}
             
             \PY{k}{for} \PY{n}{r}\PY{p}{,}\PY{n}{pt1}\PY{p}{,}\PY{n}{pt2} \PY{o+ow}{in} \PY{n+nb}{zip}\PY{p}{(}\PY{n}{lines}\PY{p}{,}\PY{n}{pts1}\PY{p}{,}\PY{n}{pts2}\PY{p}{)}\PY{p}{:}
                 \PY{n}{x0}\PY{p}{,}\PY{n}{y0} \PY{o}{=} \PY{n+nb}{map}\PY{p}{(}\PY{n+nb}{int}\PY{p}{,} \PY{p}{[}\PY{l+m+mi}{0}\PY{p}{,} \PY{o}{\PYZhy{}}\PY{n}{r}\PY{p}{[}\PY{l+m+mi}{2}\PY{p}{]}\PY{o}{/}\PY{n}{r}\PY{p}{[}\PY{l+m+mi}{1}\PY{p}{]} \PY{p}{]}\PY{p}{)}
                 \PY{n}{x1}\PY{p}{,}\PY{n}{y1} \PY{o}{=} \PY{n+nb}{map}\PY{p}{(}\PY{n+nb}{int}\PY{p}{,} \PY{p}{[}\PY{n}{c}\PY{p}{,} \PY{o}{\PYZhy{}}\PY{p}{(}\PY{n}{r}\PY{p}{[}\PY{l+m+mi}{2}\PY{p}{]}\PY{o}{+}\PY{n}{r}\PY{p}{[}\PY{l+m+mi}{0}\PY{p}{]}\PY{o}{*}\PY{n}{c}\PY{p}{)}\PY{o}{/}\PY{n}{r}\PY{p}{[}\PY{l+m+mi}{1}\PY{p}{]} \PY{p}{]}\PY{p}{)}
                 \PY{n}{cv2}\PY{o}{.}\PY{n}{line}\PY{p}{(}\PY{n}{img\PYZus{}left}\PY{p}{,} \PY{p}{(}\PY{n}{x0}\PY{p}{,}\PY{n}{y0}\PY{p}{)}\PY{p}{,} \PY{p}{(}\PY{n}{x1}\PY{p}{,}\PY{n}{y1}\PY{p}{)}\PY{p}{,} \PY{p}{[}\PY{l+m+mi}{255}\PY{p}{,}\PY{l+m+mi}{0}\PY{p}{,}\PY{l+m+mi}{0}\PY{p}{]}\PY{p}{,}\PY{l+m+mi}{1}\PY{p}{)}
                 \PY{n}{cv2}\PY{o}{.}\PY{n}{circle}\PY{p}{(}\PY{n}{img\PYZus{}left}\PY{p}{,}\PY{n+nb}{tuple}\PY{p}{(}\PY{n}{pt1}\PY{p}{)}\PY{p}{,}\PY{l+m+mi}{5}\PY{p}{,}\PY{p}{[}\PY{l+m+mi}{0}\PY{p}{,}\PY{l+m+mi}{255}\PY{p}{,}\PY{l+m+mi}{0}\PY{p}{]}\PY{p}{,}\PY{o}{\PYZhy{}}\PY{l+m+mi}{1}\PY{p}{)}
                 \PY{n}{cv2}\PY{o}{.}\PY{n}{circle}\PY{p}{(}\PY{n}{img\PYZus{}right}\PY{p}{,}\PY{n+nb}{tuple}\PY{p}{(}\PY{n}{pt2}\PY{p}{)}\PY{p}{,}\PY{l+m+mi}{5}\PY{p}{,}\PY{p}{[}\PY{l+m+mi}{0}\PY{p}{,}\PY{l+m+mi}{255}\PY{p}{,}\PY{l+m+mi}{0}\PY{p}{]}\PY{p}{,}\PY{o}{\PYZhy{}}\PY{l+m+mi}{1}\PY{p}{)}
                 
             \PY{k}{return} \PY{n}{img\PYZus{}left}\PY{p}{,}\PY{n}{img\PYZus{}right}
         
         \PY{c}{\PYZsh{} we choose the indices manually to ensure maximum separation between the epipolar lines}
         \PY{c}{\PYZsh{} sample\PYZus{}indices = [99, 83, 53, 131, 162,  0,  18]}
         \PY{n}{sample\PYZus{}indices} \PY{o}{=} \PY{n}{np}\PY{o}{.}\PY{n}{random}\PY{o}{.}\PY{n}{random\PYZus{}integers}\PY{p}{(}\PY{l+m+mi}{0}\PY{p}{,}\PY{n+nb}{len}\PY{p}{(}\PY{n}{pts2}\PY{p}{)}\PY{o}{\PYZhy{}}\PY{l+m+mi}{1}\PY{p}{,}\PY{l+m+mi}{7}\PY{p}{)}
         
         \PY{n}{lines1} \PY{o}{=} \PY{n}{cv2}\PY{o}{.}\PY{n}{computeCorrespondEpilines}\PY{p}{(}\PY{n}{pts2}\PY{p}{[}\PY{n}{sample\PYZus{}indices}\PY{p}{]}\PY{o}{.}\PY{n}{reshape}\PY{p}{(}\PY{o}{\PYZhy{}}\PY{l+m+mi}{1}\PY{p}{,}\PY{l+m+mi}{1}\PY{p}{,}\PY{l+m+mi}{2}\PY{p}{)}\PY{p}{,} \PY{l+m+mi}{2}\PY{p}{,} \PY{n}{F}\PY{p}{)}
         \PY{n}{lines1} \PY{o}{=} \PY{n}{lines1}\PY{o}{.}\PY{n}{reshape}\PY{p}{(}\PY{o}{\PYZhy{}}\PY{l+m+mi}{1}\PY{p}{,}\PY{l+m+mi}{3}\PY{p}{)}
         \PY{n}{img1e\PYZus{}1}\PY{p}{,}\PY{n}{img1e\PYZus{}2} \PY{o}{=} \PY{n}{drawlines}\PY{p}{(}\PY{n}{img1}\PY{p}{,}\PY{n}{img2}\PY{p}{,}\PY{n}{lines1}\PY{p}{,}\PY{n}{pts1}\PY{p}{[}\PY{n}{sample\PYZus{}indices}\PY{p}{]}\PY{p}{,}\PY{n}{pts2}\PY{p}{[}\PY{n}{sample\PYZus{}indices}\PY{p}{]}\PY{p}{)}
          
         \PY{n}{lines2} \PY{o}{=} \PY{n}{cv2}\PY{o}{.}\PY{n}{computeCorrespondEpilines}\PY{p}{(}\PY{n}{pts1}\PY{p}{[}\PY{n}{sample\PYZus{}indices}\PY{p}{]}\PY{o}{.}\PY{n}{reshape}\PY{p}{(}\PY{o}{\PYZhy{}}\PY{l+m+mi}{1}\PY{p}{,}\PY{l+m+mi}{1}\PY{p}{,}\PY{l+m+mi}{2}\PY{p}{)}\PY{p}{,} \PY{l+m+mi}{1}\PY{p}{,} \PY{n}{F}\PY{p}{)}
         \PY{n}{lines2} \PY{o}{=} \PY{n}{lines2}\PY{o}{.}\PY{n}{reshape}\PY{p}{(}\PY{o}{\PYZhy{}}\PY{l+m+mi}{1}\PY{p}{,}\PY{l+m+mi}{3}\PY{p}{)}
         \PY{n}{img2e\PYZus{}1}\PY{p}{,}\PY{n}{img2e\PYZus{}2} \PY{o}{=} \PY{n}{drawlines}\PY{p}{(}\PY{n}{img2}\PY{p}{,}\PY{n}{img1}\PY{p}{,}\PY{n}{lines2}\PY{p}{,}\PY{n}{pts2}\PY{p}{[}\PY{n}{sample\PYZus{}indices}\PY{p}{]}\PY{p}{,}\PY{n}{pts1}\PY{p}{[}\PY{n}{sample\PYZus{}indices}\PY{p}{]}\PY{p}{)}
         
         \PY{n}{f}\PY{p}{,} \PY{n}{axes} \PY{o}{=} \PY{n}{plt}\PY{o}{.}\PY{n}{subplots}\PY{p}{(}\PY{n}{ncols}\PY{o}{=}\PY{l+m+mi}{2}\PY{p}{,} \PY{n}{figsize}\PY{o}{=}\PY{p}{(}\PY{l+m+mi}{15}\PY{p}{,}\PY{l+m+mi}{15}\PY{p}{)}\PY{p}{)}
         \PY{n}{axes}\PY{p}{[}\PY{l+m+mi}{0}\PY{p}{]}\PY{o}{.}\PY{n}{imshow}\PY{p}{(}\PY{n}{img1e\PYZus{}1}\PY{p}{)}
         \PY{n}{axes}\PY{p}{[}\PY{l+m+mi}{0}\PY{p}{]}\PY{o}{.}\PY{n}{set\PYZus{}title}\PY{p}{(}\PY{l+s}{\PYZsq{}}\PY{l+s}{Chapel Left \PYZhy{} Epipolar Lines}\PY{l+s}{\PYZsq{}}\PY{p}{,}\PY{n}{fontsize}\PY{o}{=}\PY{l+m+mi}{20}\PY{p}{)}
         \PY{n}{axes}\PY{p}{[}\PY{l+m+mi}{1}\PY{p}{]}\PY{o}{.}\PY{n}{imshow}\PY{p}{(}\PY{n}{img2e\PYZus{}1}\PY{p}{)}
         \PY{n}{axes}\PY{p}{[}\PY{l+m+mi}{1}\PY{p}{]}\PY{o}{.}\PY{n}{set\PYZus{}title}\PY{p}{(}\PY{l+s}{\PYZsq{}}\PY{l+s}{Chapel Right \PYZhy{} Epipolar Lines}\PY{l+s}{\PYZsq{}}\PY{p}{,}\PY{n}{fontsize}\PY{o}{=}\PY{l+m+mi}{20}\PY{p}{)}
         \PY{n}{plt}\PY{o}{.}\PY{n}{show}\PY{p}{(}\PY{p}{)}
\end{Verbatim}

    \begin{center}
    \adjustimage{max size={0.9\linewidth}{0.9\paperheight}}{HW4_files/HW4_25_0.png}
    \end{center}
    { \hspace*{\fill} \\}
    

    % Add a bibliography block to the postdoc
    
    
    
    \end{document}
